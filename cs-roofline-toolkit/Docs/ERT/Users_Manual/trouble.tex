This section contains a list of suggestions from the developers and testers
of the ERT that have helped us get the ERT working when it hasn't
successfully run to completion.  Hopefully, they will be helpful if you need
to troubleshoot problems running the ERT with on your own machines with your 
own configuration files.

The easiest step, which should probably be your first step, it do get the most
output possible from the ERT.  To do this use the ``-{}-verbose=2'' option.
This will cause the ERT to print all the commands it is attempting to execute -
including the final command before the ERT failed.

Often, simply looking at the last command will help identify the problem.  If
not, look at the previous commands to see if they seem to be constructed
correctly to build, link, and run the ERT micro-kernel.

If this doesn't indicate why things failed, you can try to run the last
command interactively and see what happens.  This sometimes generates
errors/warnings that weren't shown when the command was run within the ERT.

If the last command is the running of the micro-kernel, all output (and
errors) will be redirected to a file shown on the command output.  There will
be a rather long path to the file and the file will have the name ``try.\#''
where ``\#'' is a number.  Look in this file for any and all output resulting
from running the micro-kernel - including error messages.

If you are still having a problem after trying these suggestions or if it
isn't clear how to follow some of them, feel free to contact us using the
contact information on the WWW site where you obtained the Roofline Toolkit.
We can then work with you to get the ERT running for you.
