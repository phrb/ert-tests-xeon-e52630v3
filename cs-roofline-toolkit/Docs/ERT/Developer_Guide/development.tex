This chapter gives the details of how the Empirical Roofline Tool will be
developed, tested, and released. This includes incorporating suggests and
general feedback at all stages.

Much of the initial groundwork has been completed.  We have experimented with
a number of microkernels on a variety of machines.  From these experiments,
we’ve been able to determine which codes give accurate bandwidth and GFLOP
values which can be used to generate Roofline graphs for each machine.

In addition, we’ve been able to structure the microkernels so they can be run
with combinations of MPI processes, OpenMP threads, and GPU threads.  This
allows us to characterize the hardware with a variety of different forms and
combinations of parallelism.

Other variations include the use of different compilers and compiler flags,
the use of difference runtime environments and environment variables, and
microkernels that express possible parallelism more explicitly.

For a given experiment, the current microkernels are run with a wide range of
array sizes and a wide range of trials per array size.  This allows the
benchmarks to map out a fairly large parameter space for each microkernel.
Currently, all this data is output for subsequent postprocessing.

The postprocessing consists for averaging multiple runs and extracting
bandwidth or GFLOP numbers.  In the case of bandwidth number, there are
usually a number of plateaus which correspo0nd to the array fitting in each of
the various portions of the memory hierarchy.

\section{Next Steps}
The next steps that need to be completed are:

\begin{enumerate}

\vspace{-0.1in}
\item{Take the experimental codes that have been successful on all the machine
      architectures and use them to design a single framework that will be
      used by the Empirical Roofline Tool to generate bandwidth and GFLOP
      values.}

\vspace{-0.1in}
\item{Define what raw data will be output by the unified framework in (1),
      what postprocessing will be applied to the raw data, what the processed
      data will be, and how it will be represented/externalized.}

\vspace{-0.1in}
\item{Determine what parameters will be automatically configured and which
      will be specified by the user.}

\vspace{-0.1in}
\item{Develop the configuration system for the parameters that will be
      automatically set such that it will work on all the target systems.}

\vspace{-0.1in}
\item{Develop a (simple) UI to gather the parameters that will be specified
      by the user.}

\vspace{-0.1in}
\item{Test and debug the entire system on the target machines:  Hopper, Edison,
      Mira, Titan (CPU), Babbage, Dirac, Titan (CPU+GPU).}

\vspace{-0.1in}
\item{Make the Empirical Roofline Tool available to other members of the
      Roofline Toolkit effort along with the results gathered in (6).}

\vspace{-0.1in}
\item{Based on what is learned and the data collected, expand the set of
      microkernels, vary the data access patterns, and expand the unified
      framework to collect a larger range of data spanning more of the
      multidimensional problem space.}

\vspace{-0.1in}
\item{Expand the configuration and testing to as many architectures and systems
      as possible.}

\vspace{-0.1in}
\item{Fix and update the tool as needed, document, package, and release.}

\end{enumerate}

\section{Schedule}
Here is the proposed schedule for the development outlined above:

\begin{tabular}{|L{0.58\linewidth}|C{0.16\linewidth}|C{0.16\linewidth}|}
\hline
\multicolumn{1}{|c|}{Task}                                        & Scheduled Completion Date & Actual Completion Date \\
\hhline{|=:=:=|}
Basic investigation and experiments                               & End of Summer 2014        & End of Summer 2014 \\
\hhline{|=:=:=|}
Define data output format for all data - raw and processed        & \multirow{2}{*}{Sept. 8}  & \multirow{2}{*}{Sept. 29} \\
\cline{1-1}
Determine automatic and user specified parameters                 &                           & \\
\hhline{|=:=:=|}
Understand and combine experimental codes                         & Sept. 8                   & Sept. 29 \\
\hhline{|=:=:=|}
Develop the configuration system                                  & \multirow{2}{*}{Oct. 1}   & \multirow{2}{*}{Oct. 8} \\
\cline{1-1}
Develop a (simple) UI to gather user parameters                   &                           & \\
\hhline{|=:=:=|}
Test and debug on target architectures                            & \multirow{2}{*}{Nov. 1}   & \multirow{2}{*}{N/A} \\
\cline{1-1}
Release Empirical Roofline Tool internally                        &                           & \\
\hhline{|=:=:=|}
Expand coverage of microkernels, data access patterns, etc.       & \multirow{2}{*}{Dec. 1}   & \multirow{2}{*}{N/A} \\
\cline{1-1}
Expand configuration and testing                                  &                           & \\
\hhline{|=:=:=|}
Fix and update the tool as needed, document, package, and release & Jan. 1                    & N/A \\
\hline
\end{tabular}

\section{SC 2014}

For SC 2014 the following needs to be completed.  Text in gray has been
completed:

\begin{enumerate}

\vspace{-0.1in}
\item{Minimal Requirements:}

\begin{itemize}

\vspace{-0.1in}
\item{{\color{Gray}The ERT runs on Edison using flat MPI and produces the
expected roofline (which may not be optimal).}}

\item{{\color{Gray}The ERT runs on a Linux SMP ``node'' using pure OpenMP}
and produces the expected roofline.}

\item{{\color{Gray}A basic website is set up with some information about
the ERT including a user's manual and code that can be downloaded.}}

\item{{\color{Gray}Include in the user's manual appendices the rooflines
we've generated for various architectures/machines.}}

\item{Include in the user's manaul examples of running the ERT, the plots
it produces, etc.}

\item{``Spec'' values for bandwidth and gflop rate are added to the
configuration file and propagated through to the final JSON output.  We may
provide a capability/option of plotting them on the roofline.}

{\color{Gray}
\item{Add the bandwidth and gflop rate numbers to the roofline.}
}

\end{itemize}

\vspace{-0.1in}
\item{Second Level Requirements:}

\begin{itemize}

\vspace{-0.1in}
\item{The ERT runs on Edison with MPI+OpenMP over a range of MPI processes and
threads and produces a near optimal roofline.}

\item{Add a working set size or a range of sizes to the bandwidth and gflop
rate labels on the roofline.}

\end{itemize}

\vspace{-0.1in}
\item{Third Level Requirements:}

\begin{itemize}

\vspace{-0.1in}
\item{The ERT runs on Babbage in native mode using the MIC with pure OpenMP
and gets the expected roofline.}

\end{itemize}

\vspace{-0.1in}
\item{Fourth Level Requirements:}

\begin{itemize}

\vspace{-0.1in}
\item{The ERT runs on Mira and gets the expected roofline.}

\end{itemize}

\end{enumerate}

\section{List Of Work To Do}

This is a list of work that could be done to improve the ERT.  The work has
been divided into several sections to help organize the tasks.  Of course,
some work spans multiple sections and some falls in no specific section.

For the former, one section was chosen.  For the latter, there is the
``Miscellaneous'' section.  The goal is to make new specific sections as
needed when any section gets too large.

Also, completed work will not be deleted.  Initially, it will be ``grayed
out'' in place and, eventually, it will be move to the end of the section or
to a separate section for completed work.

\subsection{Internal Tool Development}

\subsection{Additional Tool Functionality And Options}

\subsection{Validation And Verification Of The Tool}

\subsection{Documentation}

\subsection{Releasing The Tool}
